%template by Marcel Neunhoeffer & Sebastian Sternberg - Uni of Mannheim

\documentclass[a4paper, 12pt]{article}  %khai bao document class

% set up margin
\usepackage[top = 2.5cm, bottom = 2.5cm, left = 2.5cm, right = 2.5cm]{geometry} 
% \usepackage[utf8]{inputenc}  % language encoder
\usepackage[utf8]{vietnam}  % vietnamese language setting
\usepackage{multirow} % Multirow is for tables with multiple rows within one cell.
\usepackage{booktabs} % For even nicer tables.
\usepackage{graphicx}
\usepackage{ulem} % for underlined format
% packages for advanced math eqn and symbols
\usepackage{amsmath}
\usepackage{amssymb}
\usepackage{amsthm}
\usepackage{enumitem} % for itemize

% set indent of new paragraph to 0
\usepackage{setspace}
\setlength{\parindent}{0.5in}
\onehalfspacing

% Package to place figures where you want them.
\usepackage{float}

% The fancyhdr package let's us create nice headers.
\usepackage{fancyhdr}

%%Make header and footer
\pagestyle{fancy} % With this command we can customize the header style.
\fancyhf{} % This makes sure we do not have other information in our header or footer.

% Header
\lhead{\footnotesize Machine Learning 1: Homework 1}% \lhead puts text in the top left corner. \footnotesize sets our font to a smaller size.
\rhead{\footnotesize Nguyen Anh Tu @DSEB 62}

% Similar commands work for the footer (\lfoot, \cfoot and \rfoot).
% We want to put our page number in the center.
\cfoot{\footnotesize \thepage} 

%%%%%%%%%%%%%%%%%%%%%%%%%% 
\begin{document}

% Title section of the document
\thispagestyle{empty} % This command disables the header on the first page. 

\begin{tabular}{p{12.5cm}} % This is a simple tabular environment to align your text nicely 
{\large \bf National Economics University, Vietnam} \\
Faculty of Mathematics Economics \\ Data Science in Economics and Business  \\ Machine Learning 1\\
\hline % \hline produces horizontal lines.
\\
\end{tabular} % Our tabular environment ends here.

\vspace*{0.3cm} % Now we want to add some vertical space in between the line and our title.

\begin{center} % Everything within the center environment is centered.
	{\Large \bf Homework Week 1} % <---- Don't forget to put in the right number
	\vspace{2mm}
	
	{\bf Student: Nguyễn Anh Tú - ID: 11207333} % <---- Fill in your names here!
\end{center}  

\vspace{0.4cm}

%%%%%%%%%%%%%%%% Problem 1:
\section{Problem 1}

Consider the following bivariate distribution \(p(x, y)\) of two discrete random variables $X$ and $Y$
\begin{center}
    \includegraphics[]{problem1.png}
\end{center}

Compute:
\begin{enumerate}[label=(\alph*)]
    \item Compute the marginal distributions $p(x)$ and $p(y)$
    \item Compute the conditional distribution $p(x|Y=y_1)$ and $p(x|Y=y_3)$
\end{enumerate}

%%%% Solution
\textbf{Solution.}
\begin{enumerate}[label=(\alph*)]
    \item Formula for marginal distribution \( \displaystyle p(X = x_i) = \sum_j p(x_i, y_j)\) \\
    Thus marginal distributions of $p(x)$ are:
    \begin{align*}
    p(X = x_1) & = p(x_1, y_1) + p(x_1, y_2) + p(x_1, y_3)\\
    & = 0.01 + 0.05 + 0.1 = 0.16 \\
    p(X = x_2) &= 0.02 + 0.1 + 0.05 = 0.17 \\
    p(X = x_3) &= 0.03 + 0.05 + 0.03 = 0.11 \\
    p(X = x_4) &= 0.1 + 0.07 + 0.05 = 0.22 \\
    p(X = x_5) &= 0.1 + 0.2 + 0.04 = 0.34
    \end{align*}
    
    Similarly, marginal distributions of $p(y)$ are:
    \begin{align*}
    p(Y = y_1) &= p(x_1, y_1) + p(x_2, y_1) + p(x_3, y_1) + p(x_4, y_1) + p(x_5, y_1)\\
    & = 0.01 + 0.02 + 0.03 + 0.1 + 0.1 = 0.26 \\
    p(Y = y_2) &= 0.05 + 0.1 + 0.05 + 0.07 + 0.2 = 0.47 \\
    p(Y = y_3) &= 0.1 + 0.05 + 0.03 + 0.05 + 0.04 = 0.27
    \end{align*}
    \item Formula for the conditional distribution 
    $ \displaystyle p(x|Y = y_i) = \frac{p(x, y_i)}{p(Y = y_i)}$ 
    
    Thus conditional distributions of $p(x|Y=y_1)$ are:
    \begin{align*}
         p(x = x_1|Y = y_1) &= \frac{p(x_1, y_1)}{p(Y = y_1)} = \frac{0.01}{0.26} \approx 0.038 \\
         p(x = x_2|Y = y_1) &= 0.02 \div 0.26 \approx 0.077 \\
         p(x = x_3|Y = y_1) &= 0.03 \div 0.26 \approx 0.115 \\
         p(x = x_4|Y = y_1) &= 0.1 \div 0.26 \approx 0.385 \\
         p(x = x_5|Y = y_1) &= 0.1 \div 0.26 \approx 0.385
    \end{align*}
    
    Similarly, conditional distributions of $p(x|Y=y_3)$ are:
        \begin{align*}
         p(x = x_1|Y = y_3) &= \frac{p(x_1, y_3)}{p(Y = y_3)} = \frac{0.1}{0.27} \approx 0.37 \\
         p(x = x_2|Y = y_3) &= 0.05 \div 0.27 \approx 0.185 \\
         p(x = x_3|Y = y_3) &= 0.03 \div 0.27 \approx 0.111 \\
         p(x = x_4|Y = y_3) &= 0.05 \div 0.27 \approx 0.185 \\
         p(x = x_5|Y = y_3) &= 0.04 \div 0.27 \approx 0.148
    \end{align*}
\end{enumerate}

%%%%%%%%%%%%%%%%%%%%% Problem 2:
\section{Problem 2}

Consider two random variables $x$, $y$ with joint distribution $p(x, y)$. Show that:
\[E_X [X] = E_Y [E_X [x|y]]\]

Here, $E_X [x|y]$ denotes the expected value of $x$ under the conditional distribution $p(x|y)$. 

\textbf{Solution.}
Suppose that $X$ and $Y$ are two random discrete variables.

Since $E_X [x|y]$ denotes the expected value of $x$ under the conditional distribution $p(x|y)$, we can write the formula $\displaystyle E_X [x|y] = \sum_{x} x \cdot p(x | y)$ 

The value of $\sum_{x} x \cdot p(x | y)$ will change for each value of $y$, therefore distribution of $E_X [x|y]$ values is also the distribution of $y$ values. Then:
\begin{align*}
    E_Y [E_X [x | y]] &= E_Y [\sum_{x} x \cdot p(x | y)] \\
    &= \sum_{y} p(y) \sum_{x} x \cdot p(x | y) \\
    &= \sum_{y} \sum_{x} p(y) \cdot p(x|y) \cdot x \\
    &= \sum_{x} x \sum_{y} p(y) \cdot p(x|y) \\
    &= \sum_{x} x \sum_{y} p(x, y)
\end{align*}

By definition of marginal distribution, $\sum_{y} p(x, y) = p(x)$. Thus:
\begin{align*}
    E_Y [E_X [x | y]] &= \sum_{x} x \cdot p(x) \\
    &= E_X [X] \qed
\end{align*}

%%%%%%%%%%%% Problem 3:
\section{Problem 3}

Một cuộc điều tra cho thấy, ở 1 thành phố 20.7\% dân số dùng sản phẩmX ,50\% dùng loại sản phẩm Y và trong những người dùng Y thì 36.5\% dùng X. Phỏng vấn ngẫu nhiên một người dân trong thành phố đó, tính xác xuất đề người ấy:

\begin{enumerate}[label=(\alph*)]
    \item Dùng cả X và Y.
    \item Dùng Y, và biết rằng người đó không dùng X.
\end{enumerate}
\textbf{Solution.}

Gọi X là biến cố người được hỏi sử dụng sản phẩm X.

Y là biến cố người đó sử dụng sản phẩm Y.

Như vậy, theo đề bài ta có:
\[p(X) = 0.207 \]
\[p(Y) = 0.5 \]
\[p(X | Y) = 0.365\]

\begin{enumerate}[label=(\alph*)]
    \item Xác suất người được hỏi sử dụng cả X và Y là:
    \[p(X, Y) = p(X|Y) \cdot p(Y) = 0.365 \cdot 0.5 = 0.1825\]
    \item Áp dụng định lý Bayes, xác suất người được hỏi dùng Y, biết rằng người đó không dùng X có thể được viết như sau:
    \begin{align*}
    p(Y|\overline{X}) &= \frac{p(\overline{X}|Y) \cdot p(Y)}{p(\overline{X})}\\
    &= \frac{(1 - p(X|Y)) p(Y)}{1 - p(X)} \\
    &= \frac{(1 - 0.365)\cdot 0.5}{1 - 0.207} \approx 0.4
    \end{align*}
\end{enumerate}


%%%%%%%%%%%%%%%% Problem 4
\section{Problem 4} 

Prove the relationship: $V_X = E_X [x^2] - (E_X [X])^2$ , which relates the standard definition of the variance to the raw-score expression for the variance

\textbf{Solution.} 
\begin{align*}
    V_X &=  E_X[(X - E_X [X])^2]\\
    &= E_X [X^2 - 2X E_X [X] + (E_X [X])^2] \\
    &= E_X [X^2] - 2E_X [X \cdot E_X [X]] + (E_X [X])^2
\end{align*}

By definition $E_X [X] = \sum x p(x)$ or $E_X [X] = \int x f(x) dx$, for random variable $X$, $E_X[X]$ is just a constant number, then $E[E_X [X]] = E_X [X]$. Therefore:
\begin{align*}
    V_X &= E_X[X^2] - 2E_X [X] \cdot E_X [X] + (E_X [X])^2 \\
    &= E_X[X^2] - 2 (E_X [X])^2 + (E_X [X])^2 \\
    &= E_X [X^2] - (E_X [X])^2   \qed
\end{align*}

%%%%%%%%%%%%% Problem 5
\section{Problem 5}
Giả sử bạn đứng trước ba ô cửa mà đằng sau nó là một trong hai thứ: con dê hoặc một chiếc xe hơi giá trị. Bạn mong muốn mở trúng ô cửa có chiếc xe để được nhận nó (nếu mở trúng ô cửa có dê thì bạn phải rinh nó về nhà). Monty yêu cầu bạn chọn một trong các ô cửa. Dĩ nhiên bạn chọn một cách “hú họa” tại xác suất lúc này để nhận xe hơi ở mỗi ô cửa đều là $\frac{1}{3}$ . Giả sử bạn chọn ô cửa số 1. Monty sẽ giúp bạn LOẠI TRỪ 1 ĐÁP ÁN SAI bằng cách mở một ô cửa có dê trong hai ô cửa còn lại (dĩ nhiên ông ta đã biết mỗi ô cửa có gì). Sau đó bạn được lựa chọn LẦN HAI: Giữ nguyên ô cửa ban đầu hay đổi sang ô cửa còn lại chưa được lật mở?

\textbf{Solution.} 

\textbf{Cách 1}
Không mất tính tổng quát, giả sử ban đầu người chơi chọn ô cửa thứ nhất, bảng sau sẽ mô tả kết quả win (người chơi nhận được xe) hay lose (nhận được dê) khi người chơi lựa chọn giữ/đổi lựa chọn của mình trong trường hợp xe nằm ở từng ô:

\begin{center}
  \begin{tabular}{lllll}
\multicolumn{1}{c}{\textbf{Lựa chọn}} & \multicolumn{1}{c}{\textit{Xe ở ô cửa 1}} & \multicolumn{1}{c}{\textit{Xe ở ô cửa 2}} & \textit{Xe ở ô cửa 3} & \textit{Xác suất thắng} \\ \hline
Giữ     & win       & lose      & lose           & $\frac{1}{3}$ \\
Đổi     & lose      & win       & win            & $\frac{2}{3}$ \\         
\end{tabular}  
\end{center}

Như vậy, theo bảng trên ta thấy xác suất người chơi nhận được xe ô tô là cao hơn khi lựa chọn là đổi sang một trong hai ô cửa còn lại. Vì vậy người chơi \textbf{nên đổi sang ô cửa còn lại }.
\vspace{5mm}

\textbf{Cách 2}
Không mất tính tổng quát, giả sử ban đầu người chơi chọn ô cửa thứ nhất.

Gọi A là biến cố xe nằm ở ô cửa thứ nhất.

Gọi B là biến cố một trong hai ô cửa còn lại được mở ra (và tất nhiên trong ô cửa đó sẽ có dê).

Vậy xác suất người chơi nhận được xe khi giữ nguyên lựa chọn của mình sau khi một trong hai ô cửa còn lại được mở cũng là xác suất xe nằm ở ô thứ nhất sau khi mở 1 trong 2 ô còn lại và bằng $p(A|B)$

Xác suất người chơi không nhận được xe khi giữ nguyên lựa chọn của mình sau khi mở 1 trong 2 ô cửa còn lại là $p(\overline{A} | B) = 1 - p(A|B)$

Áp dụng định lý Bayes ta có:
\[p(A|B) = \frac{p(B|A) \cdot p(A)}{p(B)}\]

Biết rằng khi xe nằm ở ô thứ nhất, Monty có thể chọn ngẫu nhiên 1 trong 2 ô còn lại để mở nên xác suất một ô được mở khi xe nằm ở ô thứ nhất $p(B|A)=\frac{1}{2}$

Xác suất xe nằm ở ô cửa thứ nhất $p(A) = \frac{1}{3}$

Xác suất một ô nào đó trong hai ô còn lại được lật mở:
\begin{align*}
    p(B)&= p(\text{ô đó được mở khi xe nằm ở ô cửa 1})\\
    &+ p(\text{ô đó chắc chắn được mở vì xe nằm ở ô còn lại})\\
    &= p(\text{xe nằm ở ô cửa 1}) \cdot p(\text{một trong hai ô còn lại được mở})\\
    & + p(\text{xe nằm ở ô cửa còn lại})\cdot p(\text{một trong hai ô cửa còn lại chắn chắn được mở})\\
    &= \frac{1}{3} \cdot \frac{1}{2} + \frac{1}{3} \cdot 1 = \frac{1}{2}
\end{align*}

Vì vậy:
\begin{align*}
    p(A|B) &= (\frac{1}{2} \cdot \frac{1}{3}) \div \frac{1}{2} = \frac{1}{3}\\
    p(\overline{A} | B) &= 1 - p(A | B) = 1 - \frac{1}{3} = \frac{2}{3}
\end{align*}

Ta thấy xác suất người chơi không nhận được xe khi giữ nguyên lựa chọn của mình sau khi mở 1 trong 2 ô cửa còn lại cao hơn, vì vậy người chơi \textbf{nên đổi sang ô còn lại}.
\end{document}
